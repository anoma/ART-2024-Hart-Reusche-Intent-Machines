\newcommand{\pubtitle}{Intent Machines}

\newcommand{\pubauthA}{Anthony Hart}
\newcommand{\pubaffilA}{a}
% \newcommand{\orcidA}{0000-0001-5477-1503}
\newcommand{\authemailA}{\{anthony,d\}@heliax.dev}
% \newcommand{\eqcontribA}{}

\newcommand{\pubauthB}{D. Reusche}
\newcommand{\pubaffilB}{a}
% \newcommand{\orcidB}{0000-0001-0000-0000}
% \newcommand{\authemailB}{jonathan@heliax.dev}
% \newcommand{\eqcontribB}{}

% \newcommand{\pubauthC}{Last Author}
% \newcommand{\pubaffilC}{a}
% \newcommand{\orcidC}{0000-0001-5477-1503}
% \newcommand{\authemailC}{mail@someinstitute.com}

% Institutions/Affiliations
\newcommand{\pubaddrA}{Heliax AG}

% Corresponding author mail
\newcommand{\pubemail}{\authemailA}

\newcommand{\pubabstract}{%
This paper outlines the abstract mathematical structures that  formalize intents and the machines that process them. We provide a hierarchy of progressively more specific structures with the goal of capturing the concept of an abstract intent machine at different levels of abstraction and use these insights to understand different methods for composing/combining/coordinating these machines.
}

% Description of the SI file, placed as a footnote
% \newcommand{\pubSI}{Electronic Supplementary Information (ESI) available:
% one PDF file with all referenced supporting information.}

% Any keywords to be displayed under the abstract
\keywords{ 
Intents\sep
Intent Machine\sep
Formalization\sep
}

% Supplementary space between title/abstract and text, if needed
% \newcommand{\pubVadj}{0pt}

% ! DO NOT REMOVE OR MODIFY !
\input{templates/ART/aux-preamble.tex}
% The preprint DOI to be used as a link in the paper

\pubdoi{10.5281/zenodo.10498993}
\history{(Received: January 22, 2024; Published: February 9, 2024; Version: February 9, 2024)}